\section{Introduction}

\subsection{Background}
\cite{feu_fas}
failure of aircraft engine is serious event, requires quick and informed actions. Many reasons for aircraft engine failure, for example bird strike or technical issues.

normal aircrafts can become a glider without engine. pilot has then to find a runway or another suitable landing field. pilot has to consider many facts like wind and gliding characteristics to choose an appropriate flight path. has to get to the runway with correct height and approaching angle. this is a though situation.

FernUni Hagen develops an emergency landing assistant (ELA) to find a good path to the next runway. Considering sink rates, wind speed/direction, different rolls/angles, etc. Also doing simulations if the calculated paths are really possible to do. But not every time a runway matching the aircraft's requirements is close enough, so it is a good thing to also evaluate emergency landing fields.


\cite{feu_elfi}
emergency landing fields are always the last option, because of high risks. Therefore looking for fields with low probability of a crash landing. Depends on many factors like inclination, vegetation and land use.

With Emergency Landing Field Identification (ELFI) it is possible to automatically determine landing fields using LIDAR elevation data. very high resolution, grid size under 1m, but still not enough to spot fences, creeks, power lines. additional use of image data allows analyzing surface conditions. Idea is, to use CNN with image segmentation.

\subsection{Objectives}

\subsection{Outline}

\newpage