\section{Introduction}
The failure of an aircraft's engine during a flight is a very serious event, even for experienced pilots. You have to make quick and informed decisions to handle the life-threatening situation properly. The motorized vehicle basically becomes a glider that's usually not able to keep flight height for a long time. With a very limited reach and a quite high sink rate, the most important task as a pilot is to find a runway close enough to your current location to perform an emergency landing.

There are various facts to be considered weather a specific runway is a viable option for an emergency landing. For example the distance to the runway is one of the more obvious parameters, but also things like wind conditions and gliding characteristics of the aircraft are important. However, the runway not only has to be within reach. Another condition is that you are able to approach the runway in a correct angle and with the correct altitude, either with a direct path or with some additional holding patterns. All in all, a good pilot has to take all these things into account while being in such a stressing situation.

It is no surprise that nowadays there are companies and researchers working on tools to support the decision-making process in many ways. One project called Emergency Landing Assistant (ELA) is developed at the FernUniversität Hagen in Germany~\cite{feu_fas}. The ELA uses various parameters like sink rates in both straight and curved flight paths, wind speed and direction and also different settings like flap positions. With that it calculates a recommended flight path to a nearby runway which is updated in real-time according to situational changes. Because of the precise simulation of real-world physics, this path tends to be much more accurate than a pilot's assessment, which is mostly based on experience and gut feeling.

Sometimes there is just no suitable runway that happens to be in reach for the emergency landing. As a last option to get down without crashing, the pilot has to find a spot in the landscape where he can safely touch down the airplane. Because this is such a high risk maneuver, it is only considered if there is really no other option. To reduce the risk of a crash it is essential to find an flat area with few obstacles. For that case, the ELA contains a module called Emergency Landing Field Identification (ELFI), which assists in finding zones suitable for an emergency landing in the wild~\cite{feu_elfi}. This thesis will work on improving said module using deep learning techniques with a publicly available dataset.

\subsection{Background}
An early version of ELFI used only LIDAR elevation data to detect emergency landing fields. As stated in \cite{feu_elfi} they used high resolution height maps with an edge length of under 1m in the grid. A big drawback with elevation data is that you cannot rely on the data in every case. For example, water surfaces like lakes and large rivers are found to be perfect areas for an emergency landing. But in reality water has a much higher risk to land on compared to solid ground. Furthermore, even high resolution height maps usually miss obstructions like power lines or small creeks. In the end this approach can give suggestions for considerable emergency landing areas, but the pilot still has to filter those results very closely.

\WIP{
%\cite{feu_elfi}
%emergency landing fields are always the last option, because of high risks. Therefore looking for fields with low probability of a crash landing. Depends on many factors like inclination, vegetation and land use.

%With Emergency Landing Field Identification (ELFI) it is possible to automatically determine landing fields using LIDAR elevation data. very high resolution, grid size under 1m, but still not enough to spot fences, creeks, power lines. additional use of image data allows analyzing surface conditions. Idea is, to use CNN with image segmentation.

\begin{itemize}
    \item more information required to better reflect the current situation
\end{itemize}
}


\subsection{Objectives}
The main objective of this thesis is to examine new ways of identifying and analyzing emergency landing fields with convolutional neural networks. The former methods are based exclusively on binary classification. With that, ELFI can only suggest potential areas and provide information such as inclination. To further support the pilot's decision making, this thesis investigates how to make more information available to the pilot by making use of high resolution orthographic photos. The District Government of Cologne provides a large dataset with digital orthographic imagery of Cologne and surroundings\footnote{\url{https://www.geoportal.nrw/}}.

To achieve the objective, we perform a semantic segmentation of the image data with convolutional neural networks. This process enables to distinguish between different zones like buildings, forest or water in an image. The results are saved in a spatial database and then visualized for the reader. Afterwards it is discussed how those results can be used for landing field identification and what value they have for assisting a pilot.

In a second step, all areas that are clearly not suitable for an emergency landing (like cities and forests) are dropped. For this part of the thesis, the focus is on potential landing areas to further evaluate their suitability for an emergency landing. A major point of interest is the vegetation density, which is to be analyzed using a regression model. The results are again stored in a spatial database and visualized for discussion.
% TODO: update objective for vegetation analysis

\subsection{Outline}
\WIP{
\begin{itemize}
    \item Brief summary of all chapters and their content. This section will be written at the very end of the thesis.
\end{itemize}
}

\newpage