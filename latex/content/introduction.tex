\section{Introduction}
The failure of an aircraft's engine during a flight is a very serious event, even for experienced pilots. You have to make quick and informed decisions to handle the life-threatening situation properly. The motorized vehicle basically becomes a glider that's usually not able to keep flight height for a long time. With a very limited reach and a quite high sink rate, the most important task as a pilot is to find a runway close enough to your current location to perform an emergency landing.

There are various facts to be considered weather a specific runway is a viable option for an emergency landing. For example the distance to the runway is one of the more obvious parameters, but also things like wind conditions and gliding characteristics of the aircraft are important. However, the runway not only has to be within reach. Another condition is that you are able to approach the runway in a correct angle and with the correct altitude, either with a direct path or with some additional holding patterns. All in all, a good pilot has to take all these things into account while being in such a stressing situation.

It is no surprise that nowadays there are companies and researchers working on tools to support the decision-making process in many ways. One project called Emergency Landing Assistant (ELA) is developed at the FernUniversität Hagen in Germany~\cite{feu_fas}. The ELA uses various parameters like sink rates in both straight and curved flight paths, wind speed and direction and also different settings like flap positions. With that it calculates a recommended flight path to a nearby runway which is updated in real-time according to situational changes. Because of the precise simulation of real-world physics, this path is tends to be much more accurate than a pilot's assessment, which is mostly based on experience and gut feeling.

Sometimes there is just no suitable runway that happens to be in reach for the emergency landing. As a last option to get down without crashing, the pilot has to find a spot in the landscape where he can safely touch down the airplane. Because this is such a high risk maneuver, it is only considered if there is really no other option. To reduce the risk of a crash it is essential to find an flat area with few obstacles. For that case, the ELA contains a module called Emergency Landing Field Identification (ELFI), which assists in finding zones suitable for an emergency landing in the wild~\cite{feu_elfi}. This thesis will work on improving said module using deep learning techniques with a publicly available data set.

\subsection{Background}
An early version of ELFI used only LIDAR elevation data to detect emergency landing fields. As stated in \cite{feu_elfi} they used high resolution height maps with an edge length of under 1m in the grid. A big drawback with elevation data is that you cannot rely on the data in every case. For example, water surfaces like lakes and large rivers are found to be perfect areas for an emergency landing. But in reality water has a much higher risk to land on compared to solid ground. Furthermore, even high resolution height maps usually miss obstructions like power lines or small creeks. In the end this approach can give suggestions for considerable emergency landing areas, but the pilot still has to filter those results very closely.



\cite{feu_elfi}
emergency landing fields are always the last option, because of high risks. Therefore looking for fields with low probability of a crash landing. Depends on many factors like inclination, vegetation and land use.

With Emergency Landing Field Identification (ELFI) it is possible to automatically determine landing fields using LIDAR elevation data. very high resolution, grid size under 1m, but still not enough to spot fences, creeks, power lines. additional use of image data allows analyzing surface conditions. Idea is, to use CNN with image segmentation.

\subsection{Objectives}
improve emergency landing field identification using NN models, extending existing segmentation architectures. use orthographic photos data set of Cologne with very high resolution.

First goal is to do semantic segmentation on land cover. Identify different classes of land usage like buildings, forest, agriculture. results will be visualized and compared to real situation. also saved in postgres database with postGIS extension. this allows for future use of the data. with that a first filtering of suitable emergency landing fields can be applied, because certain land areas are more suitable than others.

% TODO make sure the wording is correct (regression model)
Build on those results, a second step analyzes vegetation density. only areas that are identified as agriculture/field will be looked at, so much smaller dataset. idea is to have a regression model that maps vegetation density to floats between 0 and 1. Also save this in postGIS database. Visualize results and discuss thresholds for detecting landing fields. Different airplanes can have different requirements and limits for this.

\subsection{Outline}

\begin{itemize}
    \item Brief summary of all chapters and their content. This section will be written at the very end of the thesis
\end{itemize}

\newpage