\section{Introduction}
The total loss of thrust of an aircraft during flight is a very serious event, even for experienced pilots. One has to make quick and informed decisions to handle the life-threatening situation properly. The motorized vehicle basically becomes a glider that's usually incapable of keeping flight height for a long time. With a very limited reach and a quite high sink rate, the most important task as a pilot is to find a runway close enough to the current location to perform an emergency landing.

There are various facts to be considered whether a specific runway is a viable option for an emergency landing. For example the distance to the runway is one of the more obvious parameters, but also things like wind conditions and gliding characteristics of the aircraft are important. However, the runway not only has to be within reach. Another condition is the ability to approach the runway in a correct angle and with the correct altitude, either with a direct path or with some additional holding patterns. All in all, a good pilot has to take all these things into account while being in such a stressing situation.

It is no surprise that nowadays there are companies and researchers working on tools to support the decision-making process in many ways. One project called Emergency Landing Assistant (ELA) is developed at the FernUniversität in Hagen in Germany~\cite{feu_fas}. The ELA uses various parameters like sink rates in both straight and curved flight paths, wind speed and direction and also different settings like flap positions~\cite{glide_path20}. With that it calculates a recommended flight path to a nearby runway which is updated in real-time according to situational changes. Because of the precise simulation of real-world physics, this path tends to be much more accurate than a pilot's assessment, which is mostly based on experience and gut feeling.

Sometimes there is just no suitable runway that happens to be in reach for the emergency landing. As a last option to prevent a crash, the pilot has to find a spot in the landscape where it is safe to touch down the airplane. Because this is such a high risk maneuver, it is only considered if there is undoubtedly no other option. To reduce the risk of a crash it is essential to find a flat area without any obstacles. For that case, the ELA contains a module called Emergency Landing Field Identification (ELFI), which assists in finding zones suitable for an emergency landing in open terrain~\cite{feu_elfi}. This thesis will work on improving said module using deep learning techniques with a publicly available dataset.

\subsection{Background}
An early version of ELFI used only LIDAR elevation data to detect emergency landing fields. As stated in~\cite{feu_elfi} they used high resolution height maps with an edge length of under 1m in the grid. A big drawback with elevation data is that you cannot rely on the data in every case. For example, water surfaces like lakes and large rivers are found to be perfect areas for an emergency landing. But in reality water has a much higher risk to land on compared to solid ground.

Furthermore, even height maps with very high resolution typically miss obstructions like power lines or small creeks. In the end this approach can give suggestions for considerable emergency landing areas, but the pilot still has to filter those results very closely. Additional data like remotely sensed images can be used in order to narrow down the number of potential emergency landing fields. With image data it is possible to recognize even very small objects as well as the surface conditions in the potential areas.

The latest version of the ELFI module already makes use of this. However, the existing solutions are based on methods of traditional image processing only. With the recent success of machine learning and artificial neural networks, this topic offers encouraging opportunities to optimize the ELFI module. For that reason, also the FernUniversität in Hagen conducts research on these approaches.

\subsection{Objectives}
The main objective of this thesis is to examine new ways of identifying and analyzing emergency landing fields with convolutional neural networks. The former methods are based exclusively on binary classification. With that, ELFI can only suggest potential areas and provide information such as inclination. To further support the pilot's decision making, this thesis investigates how to make more information available to the pilot by making use of high resolution orthographic photos. The District Government of Cologne provides a large dataset with digital orthographic imagery of the federal state of North Rhine-Westphalia\footnote{For details see \url{https://www.geoportal.nrw/}.}.

To achieve the objective, we perform a semantic segmentation of the image data with convolutional neural networks. This process enables to distinguish between different zones like buildings, forest or water in an image. One of the main tasks is to investigate and evaluate certain reference architectures for the segmentation challenge. For this purpose, the reference architectures are first discussed in theory and then implemented and trained in practice. Based on their individual results, the suitability of the architectures is discussed concerning the identification of emergency landing fields.

In a second step, only those segmentation results are considered which are potentially suitable for an emergency landing (e.~g. excluding buildings). For these segments the density of the vegetation is to be analysed using spectral vegetation indices. Therefore, some indices are explored and their respective results on the given dataset are demonstrated. Finally, it is discussed how such indices can be used for the classification of vegetation density in the context of emergency landing field identification.

\clearpage