\section{Analysis of vegetation density}
\label{sec:vegetation_analysis}

\subsection{Spectral Vegetation Indices}
\WIP{
\cite{glv03}
Spectral vegetation indices (SVI) are indicator for vegetation density. based on fact, that healthy vegetation absorbs most of visible red and blue light, but reflect green and NIR light. in fact, reflections of green lights are significantly higher than other spectrums. other materials dont show this bump in green spectrum, instead show steady rise in reflectance as wavelengths increase.

SVIs compare difference between visible and NIR light. the greater the difference, the greater the amount of green vegetation in the scene. There are various indices that each combine data from multiple spectral bands to a single value indicating the vegetation health and density. Most of them use simple algebraic formulas. Some of the commonly used indices are now introduced. 
}

\subsubsection{Ratio Vegetation Index}
\WIP{
\cite{glv03}
Ratio Vegetation Index (RVI) is simply ratio between NIR light and red light (equation \ref{eq:rvi}). Values for non-vegetation area are generally around $1$, because similar reflectance of NIR and red light. index can not go below $0$ for practical reasons (can not be less than no reflectance). Higher means more vegetation, high values typically reaching up to magnitude of $30$. however, there is no upper bound.

\begin{equation}
    \text{RVI} = \frac{\text{NIR}}{\text{Red}}
    \label{eq:rvi}
\end{equation}
}

\subsubsection{Normalized Difference Vegetation Index}
\WIP{
\cite{gisg_ndvi20}
NDVI ranges from $-1$ to $1$. Negative values indicate high probability for water. Positive values imply vegetation, where higher values mean more/denser vegetation. Values around 0 just tell that there is no vegetation, but does not tell about land cover. Could be desert, rocks or urban areas.


\cite{measuring_vegetation00}
first use of vegetation analysis was driven by National Oceanic and Atmospheric Administration (NOAA), US agency focusing on conditions of the environment. They launched satellites carrying an Advanced Very-High-Resolution Radiometer (AVHRR) instrument to measure earth's reflectance in five spectral bands. Two of sensors are sensitive to wavelengths ranging from $550$ to $700$ nanometers (red light), an $730$ to $1000$ nanometers (near-infrared light). the NDVI is based on those two values. \ref{eq:ndvi} shows formula to calculate NDVI.

\begin{equation}
    \text{NDVI} = \frac{\text{NIR}-\text{Red}}{\text{NIR}+\text{Red}}
    \label{eq:ndvi}
\end{equation}
}

\subsection{Building the Neural Network}
\WIP{
\begin{itemize}
    \item show the chosen architectures of all tested models
    \item describe the idea behind the architectural decisions
    \item analyze how the NN works internally (e. g. visualize feature/confidence maps)
    \item explain the data format for the resulting database
\end{itemize}
}

\subsection{Discussion}
\WIP{
\begin{itemize}
    \item compare results to the real landscape in the dataset
    \item discuss practical use of those results for identifying emergency landing fields
\end{itemize}
}

\newpage
