\section{Conclusion}
The main objective of this thesis was to advance the identification of emergency landing fields with the given data set. This goal was achieved by a variety of approaches.

For the semantic segmentation of terrain, three renowned reference architectures have been studied thoroughly. Each architecture was analyzed and presented in detail. The architectures were tested extensively with regards to the segmentation challenge. 

The experiments proved that the W-Net architecture is not effective on the given data set. The unsupervised training approach did not generate new insights on the data and did not lead to a adequate differentiation of class segments. For that reason, it is recommended to not conduct further research on this architecture in the given context.

By contrast, both U-Net and FC-DenseNet architectures have shown great success for the class segmentation. The predictions of U-Net were more contiguous and therefore closer to the labels used for training. On the other hand, the predictions of FC-DenseNet are closer to reality, because they represent very detailed segments. 

Both architectures expose some shortcomings with respect to the underrepresented classes in the data set. It is recommended to address these issues before installing the models in a production environment. This is likely to be related to the concerns expressed about the inaccurate labels used for training. Therefore, it is recommended to employ a different set of labels for future research. Since manual labelling of the entire data set comes with very high efforts, another predefined set of labels has to be found. It is also possible to combine information from different sources for  this purpose.  

\WIP{
Another objective of the thesis was to evaluate the use of spectral vegetation indices to rate vegetation density. The goal is to use projections of SVIs for estimating suitability of areas for emergency landing. for that purpose, several SVIs have been explored. experiments show very inconsistent results. the areas where dense vegetation is predicted differ based on the index that is used. Also the real situation of vegetation is not represented in data set, so authors can only estimate it. Therefore it is hard to assess practical usability of those indices. 

Also most indices are designed and calibrated for use with very specific hardware in very specific conditions. data set in this thesis does not match those conditions. this might explain inconsistency seen across the indices. Based on those results, no clear conclusion can be drawn about the indices. Further research is required to give recommendations. 

This requires more information about actual vegetation situation. that way it can be evaluated which indices are close to real situation. also the indices which yield inaccurate projections can be dropped. For that purpose a different data set is needed. 

In summary, this thesis made significant progress to optimize identification of emergency landing fields. 
}

\newpage